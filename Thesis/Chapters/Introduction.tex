\chapter{Introduction}
CAPTCHA (Completely Automated Public Turing Test to Tell Computers and Humans Apart) is a program used to distinguish human users from bots. A bot is a malicious application that automates a task, gathering useful information about user credentials or pretending to be a human interaction with Web application. Hence the term \textit{"bot"} is an abbreviation of the words "software robot".\\
The CAPTCHAs are traditionally used in Web applications for\cite{text_audio}:
\begin{itemize}
\item{\textbf{Online Polls}\\
CAPTCHAs prevent the creation and the submission of a large number of votes, favouring a party.
}
\item{\textbf{Protecting Web Registration}\\
CAPTCHAs prevent the creation of free mail account to bot instead of human users. The goal of the use of CAPTCHAs is to remove the possibility that the hacker could take advantages from the large amount of registrations.
}
\item{\textbf{Preventing comment spam}\\
CAPTCHAs prevent the insertion of a large amount of posts made by bot on pages of social platforms or blogs.
}
\item{\textbf{Search engine bots}\\
CAPTCHAs are used to guarantee that a website should be unindexed to prevent the reading of the page through search engine bots. The CAPTCHAs are added because the html tag, used to unindex the web page, doesn't guarantee unindexing.
}
\item{\textbf{E-Ticketing}\\
CAPTCHAs prevent that a big events would sell out minutes after tickets become available. In fact ticket scalpers that make large number of ticket purchases for big events.
}
\item{\textbf{Email spam}\\
CAPTCHAs are used to verify that a human has sent the email.
}
\item{\textbf{Preventing Dictionary Attacks}\\
CAPTCHAs prevent bot to guess the password of a specific user. The hacker could guess the password, taking it from a dictionary of passwords. The use of the CAPTCHA challenge prevents the iteration of the login phase made by the bot using all the words of the dictionary. After a certain number of failures POST requests, the CAPTCHA challenge is shown to the user.}
\item{\textbf{Verifying digitized books}\\
\textbf{DA RIVEDERE, NON HO CAPITO}\\
This is a way of increasing the value of CAPTCHA as an application. An application called reCAPTCHA harnesses users responses in CAPTCHA fields to verify the contents of a scanned piece of paper. Because computers aren't always able to identify words from a digital scan, humans have to verify what a printed page says. Then it's possible for search engines to search and index the contents of a scanned document. This is how it works: The application already recognizes one of the words. If the visitor types that word into a field correctly, the application assumes the second word the user types is also correct. That second word goes into a pool of words that the application will present to other users. As each user types in a word, the application compares the word to the original answer. Eventually, the application receives enough responses to verify the word with a high degree of certainty. That word can then go into the verified pool.
}
\end{itemize}
Another useful application of CAPTCHA is the support to the authentication process. This application is going to be analysed in details in the next chapters, looking at the authentication from smartphone.\\
In Chapter \ref{StateOfArt} there will be a description of the state of art of CAPTCHA, looking at types of CAPTCHA and the related tests from which this challenge is born.
\textbf{DESCRIPTION OF THE CONTENT OF THE CHAPTERS}