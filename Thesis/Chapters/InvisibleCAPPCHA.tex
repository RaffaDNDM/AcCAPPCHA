\chapter{Invisible CAPPCHA}\label{chapter:InvisibleCAPPCHA}
The \textit{Invisible CAPPCHA} is an evolution of CAPPCHA, in terms of usability\cite{Invisible_CAPPCHA}. The main difference with respect to CAPPCHA is that the challenge isn't explicitly submitted to user but it's hidden behind the PIN authentication phase. This type of challenge works only on smartphones as its ancestor.\\
This CAPTCHA is a method developed in 2018 and based on motion side-channel information, obtained by sensors on mobile device. The main steps that this CAPTCHA follows are:
\begin{enumerate}
\item{Motion detection}
\item{Communication between Client and Server}
\end{enumerate}
In fact, the micro-movements of the device, generated by the interaction of the user with the touch-screen, are evaluated by the \textit{Secure Element} (\textit{SE}). Then credentials are shared with the remote Service Provider if the input is inserted by a human or not.\\

\section{Motion detection}
Nowadays there exists a smart card, called SIMSense, that already integrates motion sensor. The correct functioning of the CAPTCHA depends on the Secure Element that embedded the accelerometer, blocking the access to the sensor by malicious code.\\
The accelerometer detects the acceleration over the three axis in g-force units, as a sequence of vectors over time:
$$\{ A_i\}_{i=1}^{n} = \{ (a_1^x, a_1^y, a_1^z), ..., (a_n^x, a_n^y, a_n^z)\}$$



\section{Communication between Client and Server}


\subsection{Elliptic Curve Digital Signature Algorithm (ECDSA)}


\section{Security analysis}


\subsection{Threat model}


\subsection{Strength against known attacks}