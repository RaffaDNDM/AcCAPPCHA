\chapter{Development}
\begin{comment}
\chapter{AcCAPPCHA}
The main side-channel information, that can be used in the implementation of a keylogger, depends on the party that we want to attack\cite{keylogging}:
\begin{itemize}
\descItem{The user}
{these attacks are based on the exploitation of physical information related to the typing state. For example, they can use electroencephalography (EEG), motion of the wrist in the smartwatches, video with keyboard line-of-sight and WiFi signal distortion. }
\descItem{The keyboard}
{these attacks are based on analysis of signals coming from the keyboard. For example, acoustic emanations can be exploited by using external physical sensors.}
\descItem{The host}
{these attacks are based on the physical access of the attacker to the victim machine. For example, the process footprint, the CPU load and other microarchitectural analysis can be exploited in this attacks.}
\descItem{The network}
{these attacks exploit the packets exchanged in the client-server communication. For example, a network packet can be related to a keystroke revealing the key press time of the victim and the payload size of the server response.}
\end{itemize}
Analysing this possibilities and according to the the side-channel information attacks in Section \ref{chapter:SideCH} and the structure of Invisible CAPPCHA in Section \ref{chapter:InvisibleCAPPCHA}, I design AcCAPPCHA. This type of CAPTCHA exploits acoustic side-channel of microphone to implement a keylogger that ensures that Authentication phase would be performed by a human user.
\end{comment}
The development phase was divided in the following phases:
\begin{itemize}
\item{Data acquisition for the prediction of the keys}
\item{Extraction of features from acquired data}
\item{Creation and training phase of the neural network}
\item{Acquisition of password and background audio analysis}
\end{itemize}
The whole implementation was created using \texttt{Python} language.

\section{Data acquisition for the prediction of the keys}
To create a program that record audio while user type something, I created \texttt{DatasetAcquisition.py} source file containing the relative class. After instantiating an object of the class \texttt{AcquireAudio}, it applies \texttt{record} method to this instance.\\
Inside this method, two different program are run in parallel: the first one is a key-logger that is used to classify all the recorded audio files in some directories and the second one that records an audio file during keys typing. The update of private members of the class is guaranteed through the use of the mutual exclusion (mutex) management.\\
The choice of running two different tasks in parallel was given by the need of recording audio before the start and after the end of password insertion by the user. Each recorded audio can contain several audio peaks related to multiple insertion of the same key but, during the acquisition of training and test set, I record one audio file for each key pressed.\\
Hence in this particular case, the key-logger waits for the insertion of a single key by the user and then reports it to the thread that performs audio recording. This last task also closes the audio stream and stores the audio signal into a \textit{wav} file named with a progressive number. All the audio files are dynamically organized into a set of subfolders of the output directory, each one with the name of the respective typed key.\\
The recording phase was performed using directly the built-in Realtek microphone and the keyboard of my MSI GL63 8RD laptop. The names of the subfolders/labels, in which each audio file of a pressed key is inserted, are reported in Appendix \ref{chapter:keymapping}. Looking at Table , we can see that the keylogger changes its behaviour mapping each key to an ASCII string of upper or lower alphabetic characters because otherwise many keys would be mapped into invalid names of folders (for example, the key \textit{'.'} is now mapped into the label \textit{'POINT'}). In the table, there are two columns of labels: the first one related to the label seen by the key-logger, the second one related to the label assigned by me to each key. The reason why these labels differ for some entry are:
\begin{itemize}
\descItem{higher accuracy for spatial distribution of the keys on the keyboard}{for example, \textit{'INSERT'} and \textbf{'0\_INSERT'} (with Num lock on) would be mapped into \textit{'INSERT'} by the key-logger but they are considered different thanks to the final mapping;}
\descItem{improve the classification of keys made by key-logger}{for example, \textit{'ALT'} label is wrongly mapped into \textit{'SHIFT'} by key-logger.}
\descItem{solve the problem of keys mapped only by hardware}{\textit{FN} is the only key with this problem. The key-logger doesn't detect any pressed key, when \textit{FN} is inserted. Hence, I needed to typed it and then another key to be sure that recording for \textit{FN} was performed. Then I made another python script to resize the audio signal and remove the useless second peak.}
\end{itemize}
The last two reasons are very important because they highlight also the power of acoustic side-channel. If an attacker implements an high-level key-logger exploiting also microphone information, the accuracy of its software can increase very much. In fact the hacker could collect a dataset of recordings of pressed keys on the same type of the victim's keyboard and then could train a Neural Network, that will be add in its malicious code. 

