\chapter{Development of AcCAPPCHA}
\begin{comment}
\chapter{AcCAPPCHA}
The main side-channel information, that can be used in the implementation of a keylogger, depends on the party that we want to attack\cite{keylogging}:
\begin{itemize}
\descItem{The user}
{these attacks are based on the exploitation of physical information related to the typing state. For example, they can use electroencephalography (EEG), motion of the wrist in the smartwatches, video with keyboard line-of-sight and WiFi signal distortion. }
\descItem{The keyboard}
{these attacks are based on analysis of signals coming from the keyboard. For example, acoustic emanations can be exploited by using external physical sensors.}
\descItem{The host}
{these attacks are based on the physical access of the attacker to the victim machine. For example, the process footprint, the CPU load and other microarchitectural analysis can be exploited in this attacks.}
\descItem{The network}
{these attacks exploit the packets exchanged in the client-server communication. For example, a network packet can be related to a keystroke revealing the key press time of the victim and the payload size of the server response.}
\end{itemize}
Analysing this possibilities and according to the the side-channel information attacks in Section \ref{chapter:SideCH} and the structure of Invisible CAPPCHA in Section \ref{chapter:InvisibleCAPPCHA}, I design AcCAPPCHA. This type of CAPTCHA exploits acoustic side-channel of microphone to implement a keylogger that ensures that Authentication phase would be performed by a human user.
\end{comment}