\cleardoublepage
\selectlanguage{english}
\begin{abstract} %------------------------------ ABSTRACT
%\markboth{}{} % remove header
%\thispagestyle{empty}
%This is the abstract
Over the years, the web services have become part of the daily routine of everyone and a lot of private users'information is uploaded on the network. For this reason, the development of the malwares has also grown using advanced AI techniques. For many years, the CAPTCHA (Completely Automated Public Turing Test to Tell Computers and Humans Apart) has been the protection mechanism against these dangerous programs. It is based on some tasks that the user should complete to be classified as a human. However the bots can also elude by exploiting machine learning and computer vision. Invisible CAPPCHA overcomes this approach by analysing the physical nature of the user during the authentication web service. The same idea has been used to create AcCAPPCHA (Acoustic CAPPCHA) based on the microphone side-channel. The design of the scheme follows the same strategy of Invisible CAPPCHA but it also introduces deep learning techniques during the classification of the user's activity. AcCAPPCHA is also applied during the authentication and it has been tested with human users and simple bot programs. The new type of CAPPCHA opens new horizons for a future implementation on mobile devices and voice assistants.
\end{abstract}
\addcontentsline{toc}{chapter}{Abstract}

\selectlanguage{italian}
\begin{abstract}
Con il passare degli anni, i servizi web sono entrati a far parte della routine quotidiana di chiunque e molte informazioni private degli utenti sono caricate sulla rete. Per questo motivo, lo sviluppo di malware è anche cresciuto usando tecniche avanzate di Intelligenza Artificiale. Per molti anni, il CAPTCHA (Completely Automated Public Turing Test to Tell Computers and Humans Apart) è stato il meccanismo di protezione contro questi programmi pericolosi. Esso è basato su alcune attività che l'utente dovrebbe completare per essere classificato come un umano. Comunque i bot riescono ad eluderlo sfruttando machine learning e computer vision. Invisible CAPPCHA supera questo approccio analizzando la natura fisica dell'utente durate il servizio web di autenticazione. La stessa idea è stata utilizzata per realizzare AcCAPPCHA (Acoustic CAPPCHA) basato sul side-channel del microfono. La progettazione dello schema segue la stessa strategia di Invisible CAPPCHA ma introduce anche tecniche di deep learning durante la classificazione dell'attività dell'utente. AcCAPPCHA viene anche applicato durante l'autenticazione ed è testato con utenti umani e semplici bot. Il nuovo tipo di CAPPCHA apre nuovi orizzonti per una futura implementazione su dispositivi mobili e assistenti vocali.
\end{abstract}
\clearpage
\selectlanguage{english}