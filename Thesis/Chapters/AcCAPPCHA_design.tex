\begin{comment}
\chapter{AcCAPPCHA}
The main side-channel information, that can be used in the implementation of a keylogger, depends on the party that we want to attack\cite{keylogging}:
\begin{itemize}
\descItem{The user}
{these attacks are based on the exploitation of physical information related to the typing state. For example, they can use electroencephalography (EEG), motion of the wrist in the smartwatches, video with keyboard line-of-sight and WiFi signal distortion. }
\descItem{The keyboard}
{these attacks are based on analysis of signals coming from the keyboard. For example, acoustic emanations can be exploited by using external physical sensors.}
\descItem{The host}
{these attacks are based on the physical access of the attacker to the victim machine. For example, the process footprint, the CPU load and other microarchitectural analysis can be exploited in this attacks.}
\descItem{The network}
{these attacks exploit the packets exchanged in the client-server communication. For example, a network packet can be related to a keystroke revealing the key press time of the victim and the payload size of the server response.}
\end{itemize}
Analysing this possibilities and according to the the side-channel information attacks in Section \ref{chapter:SideCH} and the structure of Invisible CAPPCHA in Section \ref{chapter:InvisibleCAPPCHA}, I design AcCAPPCHA. This type of CAPTCHA exploits acoustic side-channel of microphone to implement a keylogger that ensures that Authentication phase would be performed by a human user.
\end{comment}
\chapter{Design of the CAPTCHA}
The structure and behaviour of AcCAPPCHA are similar to the ones proposed in Invisible CAPPCHA application but adding also other task to improve the efficiency. The two phases of the verification are:
\begin{itemize}
\item{Evaluation of the user activity}
\item{Comunication of the password to remote service}
\end{itemize}
In the first phase, the application exploits data from the microphone instead of using other sensors as described in AcCAPPCHA. The CAPPCHA records two audio signals: the first one created during the insertion of the password by the user and the second one created before this activity for noise evaluation. The second signal is exploited to evaluate a noise threshold usefull for the computation of amplitude peaks in the first audio. During the insertion of the password, the instant of the time when each character was typed by the user is stored. Then the first verification is performed by looking if there exists a sequence of time instants of the peaks in the first signal that matches with the time instants stored manually for each character. 
In the second phase, the username and the password of the user will be signed through ECDSA and sent by client to the authentication service if and only if the insertion was performed by a human.

\section{Time correspondence}
The correspondence between the sequence of time instants, stored looking at the software clock, and the time instants of the audio signal is performed looking for a subsequence of audio peaks. \\
First of all, the program records an audio of 1 second before asking user to insert the password. The signal will be analysed to find its maximum value, that will be used as a threshold for the identification of the peaks. In fact, the signal samples of the audio recorded during the insertion, with value higher than the last threshold, will be considered as feasible peaks.\\
These samples are then grouped in several 5 ms windows and for each of them, the application find the time instant related to the maximum value of the signal in this window. For example, given \textit{the sampling period or interval}$t_s$ and a specific window of samples: $$x = [x_t, x_{t+t_s}, ..., x_{t+\lceil \frac{10ms}{t_s}\rceil * t_s}]$$
and then we compute $t'= argmax(x)$.\\
Then we declare that there is a time correspondence if given the sequence of computed time instants relative to max value of each window $t=[t^1, t^2, ..., t^n]$ if there exists a subset of it $t^{*}$ with size equal to the length of the password and that matches with a threshold with the sequence of time instants stored during the password insertion.\\
Each key of the keyboard produces a variation of the signal, called \textit{press peak}, for a time window of about 8-10 ms\cite{keyboard_acoustic}. This signal trend can also be divided in three consecutive and distinctive areas: a touch peak, a noisy meaningless area and a hit peak. The first peak is the most significant for identification of a specific key but considering also the hit peak, the key can be estimated with higher precision.\\