\chapter{Future work}\label{chapter:Future}
In the future, AcCAPPCHA could be tested on several Operating Systems to make sure that it works correctly on all of them. The application could still be improved by collecting more audio files and increasing the accuracy of the classification approach based on the spectrograms. We can add the management of a blacklist on server side of the AcCAPPCHA to prevent Brute Force attacks and DOS attacks, as explained in \myref{Section}{Results:security}.\\
The most important work that could be added is the implementation of time correspondence approach on smartphones. We can exploit the analysis of audio signals recorded by the microphones of the mobile phones. In fact, many smartphones are now equipped with two microphones and these can be used to increase the accuracy during the research of the audio peaks.\\
Another task that could be added to the mobile version of AcCAPPCHA is the character correspondence. It could exploit the shape of the waves but also the time difference between the two audio peaks, related to the same typed character, and  recorded by the built-in microphones\cite{smartphone_acoustic}. \\
Moreover the voice assistants, like Amazon Echo and Google Home, are becoming increasingly widespread and they are equipped with several microphones. These devices record human activity continuously and only if the user says a specific keyword, they turn on speakers and reply to user. Hence, the audio signals could be exploited to develop new side-channel attacks and analyse input on physical or virtual keyboards\cite{voice_assistant}. Hence the collected information could be exploit to develop also new version of AcCAPPCHA with these devices.\\
VALIDATION SET FOR IMPROVEMENT