\chapter{Future work}\label{chapter:Future}
In the future, AcCAPPCHA could be tested on several Operating Systems to make sure that it works correctly on all of them. The application could still be improved by collecting more audio files, with different types of background noise, and increasing the accuracy of the classification, in particular for the one based on the spectrograms. The classification with the neural network technique could also be improved by adding a validation set to perform some advanced techniques and removing the noise from audio files using advanced techniques. The security of AcCAPPCHA could be improved by managing the spectrogram images from the application without the involvement of the File System.\\
It would also be very interesting to measure the dependency of the training from:
\begin{enumerate}
\item{different users typing styles;}
\item{different level of wear and tear of a same hardware.}
\end{enumerate}
The most relevant development of AcCAPPCHA, however, would be the porting to the smartphone and tablet form factor. I can leverage the analysis of audio signals recorded by the microphones of the mobile phones. In fact, many smartphones are now equipped with two microphones and these can be used to increase the accuracy during the research of the audio peaks. To classify a key press, the application could leverage the shape of the waves, discovering some general pattern of an audio peak, as done in the Asonov and Agrawal's work\cite{keyboard_acoustic}. Instead of using the previous technique, AcCAPPCHA could also leverage the time difference between the audio files of the two microphones to classify a key press\cite{smartphone_acoustic}.\\
Moreover the voice assistants, like Amazon Echo and Google Home, are becoming increasingly widespread and they are equipped with several microphones. These devices record human activity continuously and only if the user says a specific keyword, they turn on speakers and reply to user. Hence, the audio signals could be leveraged to develop new attacks and to analyse input on physical or virtual keyboards\cite{voice_assistant}. However the acoustic side-channel of the voice assistant, could also be leveraged to implement a new version of AcCAPPCHA.\\