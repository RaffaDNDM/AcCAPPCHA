\chapter{Future work}\label{chapter:Future}
In the future, AcCAPPCHA could be tested on several Operating Systems to make sure that it works correctly on all of them. The application could still be improved by collecting more audio files and increasing the accuracy of the classification approach based on the spectrograms. The classification with the neural network technique could also be improved by adding a validation set to perform some advanced technique to improve the weights of the neurons. The main work, that could be done in the future, is the implementation of the time correspondence approach on the smartphones. We can exploit the analysis of audio signals recorded by the microphones of the mobile phones. In fact, many smartphones are now equipped with two microphones and these can be used to increase the accuracy during the research of the audio peaks. In future implementations, the security of AcCAPPCHA could be improved by managing the spectrogram images from the application without the involvement of the File System.\\
Another task that could be added to the mobile version of AcCAPPCHA is the character correspondence. To classify a key press, the application could exploit the shape of the waves, discovering some general pattern of an audio peak, as done in the Asonov and Agrawal's work\cite{keyboard_acoustic}. A future implementation could also exploit the time difference between the audio files recorded by the two microphones of the smartphone\cite{smartphone_acoustic}.\\
Moreover the voice assistants, like Amazon Echo and Google Home, are becoming increasingly widespread and they are equipped with several microphones. These devices record human activity continuously and only if the user says a specific keyword, they turn on speakers and reply to user. Hence, the audio signals could be exploited to develop new attacks and to analyse input on physical or virtual keyboards\cite{voice_assistant}. However the audio side-channel of voice assistant, could also be exploited to implement a new version of AcCAPPCHA with these devices.\\