\chapter{State of Art}\label{StateOfArt}
\section{Design of CAPTCHA}
CAPTCHA takes inspiration and is related to three main elements\cite{types_CAPTCHA}:
\begin{enumerate}
\item{\textbf{Turing test}\\
it's used to determine how much a machine can think like a human. The test is made by three figures: a human examiner, an human and a machine. The examiner asks some questions to both other two figures and, after a fixed amount of time, evaluates if the two answers are different or not.\\
If they are similar w.r.t. the point of view of the examiner, the machine is an AI (Artificial Intelligence) similar to an human. The test is very important if the answers have many possibilities. 
}
\item{\textbf{Human-Computer Interaction (HCI)}\\
according to cognitive psychology studies, a human process data in a specific way and this test evaluates the interaction between humans and machines. The HCI model is divided into five levels:
\begin{itemize}
\item{task level}
\item{semantic level}
\item{syntactic level}
\item{interactive level}
\item{a level of physical devices}
\end{itemize}   
Then the obtained information is processed by:
\begin{itemize}
\item{reasoning}
\item{problem solving}
\item{skill acquisition}
\item{error}
\end{itemize}   
}
\item{\textbf{Human Interactive Proof (HIP)}\\
it's used to make differentiation between machine and human users and computer user programs. The test require a type of interaction, that is simple to be done by human instead of bot. The main goals of this type of test are:
\begin{itemize}
\item{To differentiate the humans from the computers}
\item{To differentiate a category of the humans}
\item{To differentiate a specific human from the category of humans}
\end{itemize}
HIP has the test program that is subjected to the human and the computer. As a result, only a specific group of humans can positively solve the test and then the test results can be validated by the computer.
}
\end{enumerate}
In order to guarantee a good level of security, a CAPTCHA has to satisfy the following requirements:
\begin{itemize}
\item{The solution to the CAPTCHA isn't conditional and shouldn't depend on the user's language and/or age.}
\item{The solution of the CAPTCHA must be easy for the humans and hard for the bots. Hence, humans in no longer than 30 seconds with very high success rate}
\item{The creation of the CAPTCHA must not disturb the user privacy (not linked to the user).}
\end{itemize}

\section{Traditional CAPTCHA}
The traditional CAPTCHAs are based on the knowledge and correct insertion of solution by the user. The main types of this CAPTCHAs are: 
\begin{itemize}
\item{\textbf{Arithmetic}\\
}
\item{\textbf{Audio-based}\\
}
\item{\textbf{Game-based}\\
}
\item{\textbf{Image-based}\\
}
\item{\textbf{Puzzle-based}\\
}
\item{\textbf{Text-based}\\
}
\item{\textbf{Video-based}\\
}
\end{itemize}
Some types of CAPTCHA don't destroy a session, after the correct answer is inserted by the user\cite{text_audio}. Hence, the hacker can crack following accesses using the same session id with the related solution of the challenge, after connecting to the web page of CAPTCHA. In this way the attacker can make hundreds of requests before the session expires and the previous operation must be computed again.

\begin{sidewaystable}
\centering \footnotesize
\renewcommand*\arraystretch{1.3}
\begin{tabular}{cll}
\hline
\multicolumn{1}{c}{\textbf{CAPTCHA type}} & \multicolumn{1}{c}{\textbf{Usability issues}} & \multicolumn{1}{c}{\textbf{Security}}\\
\hline
\textit{Arithmetic} & {} & {}\\
\hline
\textit{Audio-based} & {
  \begin{minipage} [t] {0.4\textwidth}
  Issues of recognition:\\
      \begin{tabitem}
        \item{Previous knowledge of English dictionary by the user.}
        \item{Some character sounds very similar to others.}
      \end{tabitem} 
  \end{minipage}
} & {
  \begin{minipage} [t] {0.4\textwidth}
  It can be broken by Automatic Speech Recognition (ASR) programs (as mentioned in \cite{improving_audio}).
  \end{minipage}
}\\
\tabularnewline
\hline
\textit{Game-based} & {} & {}\\
\hline
\textit{Image-based} & {
 \begin{minipage} [t] {0.4\textwidth}
Difficulty of identification of images caused by:\\
      \begin{tabitem}
        \item{Blur of images.}
        \item{Low vision condition.}
       \end{tabitem} 
  \end{minipage}
} & {}\\
\tabularnewline
\hline
\multirow{2}{*}{\textit{Puzzle-based}} & {It takes too much time to solve the puzzle} & {}\\
{} & {and to identify the arrangement of puzzles.} & {}\\
\hline
\textit{Text-based} & 
{
  \begin{minipage} [t] {0.4\textwidth}
	Many problems have to be solved by user:\\
      \begin{tabitem}
        \item{Multiple fonts.}
        \item{Font size.}
        \item{Blurred Letters}
        \item{Wave Motion.}
       \end{tabitem} 
  \end{minipage}
} & 
{
  \begin{minipage} [t] {0.4\textwidth}
	It can be identified by:\\
      \begin{tabitem}
        \item{OCR (Optical Character Recognition) technique}
        \item{Segmentation techniques (e.g. DECAPTCHA\cite{DECAPTCHA})}
        \item{Machine Learning and Deep Learning techniques}
       \end{tabitem} 
  \end{minipage}
}\\
\tabularnewline
\hline
\textit{Video-based} & {Issues downloading videos to find correct} & {}\\
{} & {captcha because of large size of files.} & {}\\
\hline
\end{tabular}
\end{sidewaystable}

\section{Alternatives}
This types of CAPTCHA and authentication mechanisms are far from traditional CAPTCHAs and aren't based on cognitive knowledge of the human user but on other parameters:
\begin{itemize}
\item{\textbf{Biometrics-based}\\
}
\item{\textbf{Behavioural-based}\\
}
\item{\textbf{Social media sign-in}\\
}
\end{itemize}